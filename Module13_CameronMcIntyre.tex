\documentclass[svgnames]{article}
\usepackage[utf8]{inputenc}
\usepackage{amsmath}
\usepackage{amssymb}
\usepackage{mathrsfs}
\usepackage{mathtools}
\newtheorem{mydef}{Given}
\newtheorem{mytheorem}{Theorem}
\usepackage{enumitem}
\usepackage{venndiagram}
\usepackage{smartdiagram}
\usepackage{caption}
\usepackage{subcaption}
%\usepackage[framed,numbered,autolinebreaks,useliterate]{mcode}
\usepackage{pgfplots}
%\usepackage{tkiz}
\usepackage{listings}
\definecolor{dkgreen}{rgb}{0,0.6,0}
\definecolor{gray}{rgb}{0.5,0.5,0.5}
\definecolor{mauve}{rgb}{0.58,0,0.82}
\usepackage{float}

\lstset{ %
  language=R,                     % the language of the code
  basicstyle=\footnotesize,       % the size of the fonts that are used for the code
  numbers=left,                   % where to put the line-numbers
  numberstyle=\tiny\color{gray},  % the style that is used for the line-numbers
  stepnumber=1,                   % the step between two line-numbers. If it's 1, each line
                                  % will be numbered
  numbersep=5pt,                  % how far the line-numbers are from the code
  backgroundcolor=\color{white},  % choose the background color. You must add \usepackage{color}
  showspaces=false,               % show spaces adding particular underscores
  showstringspaces=false,         % underline spaces within strings
  showtabs=false,                 % show tabs within strings adding particular underscores
  frame=single,                   % adds a frame around the code
  rulecolor=\color{black},        % if not set, the frame-color may be changed on line-breaks within not-black text (e.g. commens (green here))
  tabsize=2,                      % sets default tabsize to 2 spaces
  captionpos=b,                   % sets the caption-position to bottom
  breaklines=true,                % sets automatic line breaking
  breakatwhitespace=false,        % sets if automatic breaks should only happen at whitespace
  title=\lstname,                 % show the filename of files included with \lstinputlisting;
                                  % also try caption instead of title
  keywordstyle=\color{blue},      % keyword style
  commentstyle=\color{dkgreen},   % comment style
  stringstyle=\color{mauve},      % string literal style
  escapeinside={\%*}{*)},         % if you want to add a comment within your code
  morekeywords={*,...}            % if you want to add more keywords to the set
} 


%\renewcommand{\theenumi}{\Alph{enumi}}
\newenvironment{amatrix}[1]{%
  \left(\begin{array}{@{}*{#1}{c}|c@{}}
}{%
  \end{array}\right)
}

\newenvironment{tolerant}[1]{%
  \par\tolerance=#1\relax
}{%
  \par
}


\pgfmathdeclarefunction{gauss}{2}{%
  \pgfmathparse{1/(#2*sqrt(2*pi))*exp(-((x-#1)^2)/(2*#2^2))}%
}


\title{Statistical Methods: Homework 13}
\author{Cameron McIntyre}
\date{\today}

\begin{document}

\maketitle

\section{13.3.2}
 Recall the depth perception data described in Question 8.2.6. Use a paired t test with $\alpha = 0.05$ to compare the numbers of trials needed to learn depth perception for Mothered and Unmothered lambs.

\subsection*{Answer:}

WE USE A PAIRED 2 SIDED T TEST

\begin{lstlisting}
mothered = c(2,3,5,3,2,1,1,5,3,1,7,3,5)
> mothered = c(2,3,5,3,2,1,1,5,3,1,7,3,5)
> unmothered = c(3,11,10,5,5,4,2,7,5,4,8,12,7)
> total = cbind(mothered, unmothered, mothered-unmothered)
> 
> t.test(mothered, unmothered, paired=TRUE, alternative = "two.sided")

	Paired t-test

data:  mothered and unmothered
t = -4.5029, df = 12, p-value = 0.000723
alternative hypothesis: true difference in means is not equal to 0
95 percent confidence interval:
 -4.794047 -1.667491
sample estimates:
mean of the differences 
              -3.230769 
\end{lstlisting}

Since the p value is less than .025 $(0.000723<.025)$, We reject $H_0$ and conclude the difference are significant.

\section{13.3.6}
Let $D_1, D_2, ..., D_b$ be the within-block differences as defined in this section. 
Assume that the $D_i$ are normal with $\mu_D$ and variance $\sigma_D^2$, for $i = 1, 2,..., b$ 
Derive formula for $100(1-\alpha)$ confidence interval for $\mu_D$. 
Apply this formula to the data of Case Study 13.3.1 and construct 
a $95\%$ confidence interval for the true average hemoglobin difference (\" before walk \" and  \" after walk\").


\subsection*{Answer:}
$$P(-t_{\frac{\alpha}{2},n}<\frac{\bar{d}_i-\mu}{\frac{\sigma}{\sqrt{n}}}<t_{\frac{\alpha}{2},n})=1-\alpha$$
$$P(-t_{\frac{\alpha}{2},n}<\frac{\bar{d}-\mu}{\frac{\sigma}{\sqrt{n}}}<t_{\frac{\alpha}{2},n})=1-\alpha$$
$$P(-t_{\frac{\alpha}{2},n}*\frac{\sigma}{\sqrt{n}}<\bar{d}-\mu<t_{\frac{\alpha}{2},n}*\frac{\sigma}{\sqrt{n}})=1-\alpha$$
$$P(-t_{\frac{\alpha}{2},n}*\frac{\sigma}{\sqrt{n}}+\mu<\bar{d}<t_{\frac{\alpha}{2},n}*\frac{\sigma}{\sqrt{n}}+\mu)=1-\alpha$$
So,
$$C.I. = (\mu-t_{\frac{\alpha}{2},n}*\frac{\sigma}{\sqrt{n}}<\bar{d}<\mu+t_{\frac{\alpha}{2},n}*\frac{\sigma}{\sqrt{n}})$$

Plugging in the values from the example:
$$C.I. = (.47 - 2.2281*\frac{.813}{\sqrt{10}},.47 + 2.2281*\frac{.813}{\sqrt{10}})$$
$$C.I. = (-.1,1.04)$$


\section{14.2.6}
Analyze the Shoshoni rectangle data (Case Study 7.4.2) with a sign test. Let $\alpha = 0.05$. 


\subsection*{Answer:}
The hypotethical median should be .618 for the rectangles fo be of the golden ratio variety. 
\begin{lstlisting}
shoshani <- c( 0.693,0.662, 0.690, 0.606, 0.570,0.749, 0.672, 0.628, 0.609, 0.844, 0.654, 0.615, 0.668, 0.601, 0.576, 0.670, 0.606, 0.611, 0.553, 0.933)
> med <- sum(shoshani>.618)
> count <- length(shoshani)
> testst <- (med-count/2)/(sqrt(20/4))
> pval <- pnorm(testst)
> med
[1] 11
> count
[1] 20
> testst
[1] 0.4472136
> pval
[1] 0.6726396
> \end{lstlisting}

Since the p value is .67, we fail to reject $H_0$ and conclude that the the rectangles are similar to the golden ratio variety.
\end{document}
